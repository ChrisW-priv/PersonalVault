\documentclass{article}
\usepackage{graphicx} % Required for inserting images

\begin{document}
\begin{titlepage}
   \begin{center}
       \vspace*{1cm}

       \textbf{\Huge Signals, Systems and Modulations} \\
        \vspace{2.0cm}
        \huge{Laboratory no. 4}
         \vspace{.7cm}
       \huge{\\ \today}

        \vspace{3.0cm}
        \textsc{Laboratory group members:}
        
        \vspace{0.4cm}
        Jan Chorąży\\
         \vspace{0.4cm}
        Krzysztof Watras\\
         
        \vspace{2 cm}   
       \small{Computer Science} \\  
       \vspace{0.2cm}       
       \small{Politechnika Warszawska} 
       \date{\today}
   \end{center}
\end{titlepage} 

\section*{Introduction}
During Laboratory 4 we study the efficiency of different filters given a noisy
signal. In one time period there are 2400 readings. 

\section*{Measurement results for part 1}
We observe the signal without any filter first to establish what is a
theoretical worst number of errors. We observe an average of 552 errors in one
time period. This is equivalent to 23\% error rate that is quite terrible for
any application.

Second, we study the $R_c$ filter setup in different filter bands set. 
We get measurements the same as in table below.
\begin{table}[ht!]
    \caption{Results of $R_c$ filter setup}
    \label{tab:Rcresults}
    \begin{center}
        \begin{tabular}[c]{l|l}
            Filter band [kHz] & Amount of errors \\
            1.5 & 35 \\
            2 & 30   \\
            3 & 25   \\
            4 & 23   \\
            6 & 45   \\
        \end{tabular}
    \end{center}
\end{table}

As you can see on Table~\ref{tab:Rcresults}, the best filter band in this
setup is equal to 4[kHz], with error count of 23. This is quite ok result,
equivalent to 0.8\% error rate.

After that, we study the $2 R_c$ setup. Similarly to previous task, we
explore the best band to get the lowest error count.
Measurement are given below.
\begin{table}[ht!]
    \caption{Results of $2R_c$ filter setup}
    \label{tab:2Rcresults}
    \begin{center}
        \begin{tabular}[c]{l|l}
            Filter band [kHz] & Amount of errors \\
            1.5 & 50\\
            2 & 32\\
            3 & 18\\
            4 & 23\\
            6 & 35\\
        \end{tabular}
    \end{center}
\end{table}

As you can see on Table~\ref{tab:2Rcresults}, the best filter band in this
setup is equal to 3[kHz], with error count of 18.

Finally, we measure the signal using the optimal filter. This way, we get the
best number of errors, equal to 15 errors in one period. This is quite good,
equivalent to 0.6\% error rate.

\section*{Part 1}
Q: Compare the receivers considering their resistance to channel noise 

Compared to the result with no filter ($P_e=0.46$), almost all proposed filters
have done reasonably well. The correlator has unexpectedly done the best job of
filtering out the noise and has obtained the lowest result of $P_e=0.0125$. The
second most efficient filter was 2RC in the bandwidth of 3[kHz]. For RC the most
efficient value seemed to be 4kHz, but the measured results for 3kHz and 4[kHz]
were very close, so the value might be somewhere in between. In conclusion, the
integrator had the highest resistance to channel noise, hence the lowest $P_e$,
followed by 2RC, and then RC. 

\section*{Part 2} 
\subsection*{2.1}
Q: Compare the results, how could you describe the optimal splitting?   

\begin{table}[ht!]
    \caption{Change of probability when using different splitting methods}
    \label{tab:tab21}
    \begin{center}
        \begin{tabular}[c]{l|l|l}
            value & split using x axis & suggested\\
            $P_{01}$ & 0.01096 & 0.00096\\
            $P_{10}$ & 0.01236 & 0.00116\\
            $P_e$ & 0.02332 & 0.00212\\
            $E$ & 1 & 1\\
        \end{tabular}
    \end{center}
\end{table}

As we can see on Table~\ref{tab:tab21}, the optimal splitting is an order of magnitude better than
splitting by the x-axis. The optimal splitting is obtained by fitting a line
which tries to divide the obtained samples “by color” - leaving the red samples
on the other side of the line from the blue dots. 

\subsection*{2.2} 
Q: Is it a good idea to use different $P_0$ and $P_1$?
\begin{table}[ht!]
    \caption{Change of $P_e$ and entropy based on different $p_1, p_2$ values}
    \label{tab:tab22}
    \begin{center}
        \begin{tabular}[c]{l|l|l}
            value & $p_1$=0.8, $p_0$ =0.2 &	$p_1$=0.2, $p_0$ =0.8\\
            $P_e$ & 0,00188 & 0,00172\\
            $E$ & 0,721928 & 0,721928\\
        \end{tabular}
    \end{center}
\end{table}

No, while it might seem like a good idea because of the lower $P_e$ value
compared to an equal split of $P_1$ and $P_0$. The entropy value is noticeably
smaller, thus sending less information. 

\subsection*{2.3}
Q: Which transmission code is more resistant to channel noise and why?   

\begin{table}[ht!]
    \caption{Change of $P_e$ and entropy based on transmission code}
    \label{tab:tab22}
    \begin{center}
        \begin{tabular}[c]{l|l|l}
            value & bipolar & unipolar\\
            $P_e$ & 0,02312 & 0,07836\\
            $E$ & 1 & 1\\
        \end{tabular}
    \end{center}
\end{table}

Looking at the results it is clear, that the bipolar mode is significantly more
resistant to noise. This is because the bipolar signal uses two values of opposite signs
to encode information leading to higher $\Delta$ between 2 states being send. This means 
that for the same signal power we get much lower probability of error and thus, better
communication channel.

\subsection*{2.4} 
\begin{table}[ht!]
    \caption{Unipolar and bipolar methods for different p1 values}
    \label{tab:tab22}
    \begin{center}
        \begin{tabular}[c]{l|l|l|l|l}
            value & p1=0.8 bipolar & p1=0.2 bipolar & p1=0.8 unipolar & p1=0.2 unipolar\\
            $P_e$ & 0,01808 & 0,01796 & 0,09488 & 0,00992\\
            $E$ & 0,721928 & 0,721928 & 0,721928 & 0,721928\\
        \end{tabular}
    \end{center}
\end{table}

Comment on observed results: 

The probability of error for the bipolar mode stays relatively similar
regardless of selected values of $P_1$ and $P_0$. Meanwhile, the selection of better
$P_1$ and $P_0$ can have drastic influence over the probability of error in the
unipolar method. The most efficient of the four configurations is $P_1=0.2$ for
the unipolar method, while setting $P_1=0.8$ gives the worst overall result. 


\end{document}

