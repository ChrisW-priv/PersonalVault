\documentclass{article}
\usepackage{graphicx} % Required for inserting images

\begin{document}

\begin{titlepage}
   \begin{center}
       \vspace*{1cm}

       \textbf{\Huge Signals, Systems and Modulations} \\
        \vspace{2.0cm}
        \huge{Laboratory no. 1}
         \vspace{.7cm}
       \huge{\\ \today}

        \vspace{3.0cm}
        \textsc{Laboratory group members:}
        
        \vspace{0.4cm}
        Jan Chorąży\\
         \vspace{0.4cm}
        Krzysztof Watras\\

     
      % \includegraphics[width=1.0\textwidth]{uniwersytet.jpg}
         
        \vspace{2 cm}   
       \small{Computer Science} \\  
       \vspace{0.2cm}       
       \small{Politechnika Warszawska} 
       \date{\today}
            
   \end{center}
\end{titlepage} 

\begin{center}
\textbf{\Large  Documentation of laboratory work results }\\ 
\end{center}

\begin{table}[h!]
  \begin{center}
    \caption{}
    \label{tab:task1tab}
    \begin{tabular}{c|c|c|c} 
      \textbf{N} & \textbf{transition band} & \textbf{passband ripple} 
      & \textbf{stopband attenuation} \\
      \hline
        16 & 0.0938 & 0.1577 & -15.16 \\
        32 & 0.043  & 0.186  & -16.21 \\
        64 & 0.0234 & 0.1999 & -16.65 \\
    \end{tabular}
  \end{center}
\end{table}

\subsection*{Design of lowpass FIR filter by sampling in frequency domain}
Influence of N on transition band \\
As N increases, the transition band gets smaller.

Does passband depend much on N? What is the difference between
the minimum and maximum values of passband ripple that we have
observed? \\
Ans2

Does the stopband attenuation depend much on N? What is the difference between 
the minimum and maximum values of stopband attenuation that we have observed? \\
Ans3

Number of DFT points needed to be used to obtain transition band < 0.05 \\
Ans4

How do zeros of the transfer function influence frequency response of the filter \\
Ans5

\subsection*{Design of a lowpass FIR filter by windowing in time domain}

\begin{table}[h!]
  \begin{center}
    \caption{}
    \label{tab:task1tab}
    \begin{tabular}{c|c|c|c} 
      \textbf{N} & \textbf{transition band} & \textbf{passband ripple} 
      & \textbf{stopband attenuation} \\
      \hline
        16 & 0.0938 & 0.1577 & -15.16 \\
        32 & 0.043  & 0.186  & -16.21 \\
        64 & 0.0234 & 0.1999 & -16.65 \\
    \end{tabular}
  \end{center}
\end{table}

\subsection*{Comparison of results with results of sampling in frequency domain}

Is it possible to obtain the stopband attenuation $>$ 30dB? \\
Ans6

\begin{table}[h!]
  \begin{center}
    \caption{}
    \label{tab:task1tab}
    \begin{tabular}{c|c|c} 
      \textbf{window} & \textbf{transition band} & \textbf{stopband attenuation} \\
      \hline
        rectangular & 0.0312 & -21.46 \\
        Hamming     & 0.1094 & -52.66 \\
        Blackman    & 0.1445 & -76.66 \\
    \end{tabular}
  \end{center}
\end{table}

What is the influence of window shape on transition band? \\
Ans7

What is the influence of window shape on stopband attenuation \\
Ans8

Window and its lenght N to obtain the stopband attenuation > 70 dB and 
transition band < 0.05. \\
Ans9

\subsection*{Observation of a lowpass IIR Butterworth filter}

Are there ripples in passband and in stopband? \\
Ans10

Where are zeros ot the transfer function? \\
Ans11

Where are the poles? \\
Ans12

What is the influence of the cutoff frequency on zeros and poles? \\
Ans13

\subsection*{Design of a lowpass IIR Butterworth filter}
\begin{table}[h!]
  \begin{center}
    \caption{Simulation of Butterworth filter}
    \label{tab:Frequency vs SNRbd}
    \begin{tabular}{c|c} 
      \textbf{Number of zeros and poles} & \textbf{transition band} \\
      \hline
      8 &  0.289 \\
      16 & 0.1582 \\
      32 & 0.084 \\
    \end{tabular}
  \end{center}
\end{table}

\subsection*{Design of a lowpass IIR eliptic filter}

Do we observe ripples in passband and in stopband?\\
Ans13

Where are zeros of transfer function? \\
Ans14

Where are the poles? \\
Ans15

\begin{table}[h!]
  \begin{center}
    \caption{}
    \label{tab:task1tab}
    \begin{tabular}{c|c|c} 
      \textbf{requred transition band} & \textbf{0.05} & \textbf{0.005} \\
      \hline
        obtained filter order & 13 & 20 \\
        observed transition band     & 0.1094 & -52.66 \\
        observed passband ripple    & 0.1445 & -76.66 \\
    \end{tabular}
  \end{center}
\end{table}

Are all requirements fulfield? \\
Ans16

\end{document}

