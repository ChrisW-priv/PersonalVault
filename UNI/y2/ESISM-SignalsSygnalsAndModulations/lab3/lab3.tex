\documentclass{article}
\usepackage{graphicx} % Required for inserting images

\begin{document}

\begin{titlepage}
   \begin{center}
       \vspace*{1cm}

       \textbf{\Huge Signals, Systems and Modulations} \\
        \vspace{2.0cm}
        \huge{Laboratory no. 3}
         \vspace{.7cm}
       \huge{\\ \today}

        \vspace{3.0cm}
        \textsc{Laboratory group members:}
        
        \vspace{0.4cm}
        Jan Chorąży\\
         \vspace{0.4cm}
        Krzysztof Watras\\

     
      % \includegraphics[width=1.0\textwidth]{uniwersytet.jpg}
         
        \vspace{2 cm}   
       \small{Computer Science} \\  
       \vspace{0.2cm}       
       \small{Politechnika Warszawska} 
       \date{\today}
            
   \end{center}
\end{titlepage} 

\begin{center}
\textbf{\Large  Documentation of laboratory work results }\\ 
\end{center}

\begin{table}[ht!]
  \begin{center}
      \caption{Improvement of SNR due to adaptive filtering}
    \label{tab:task1tab}
    \begin{tabular}{c|c} 
      \textbf{Adaptation speed} & \textbf{SNR improvement} \\
      \hline
        0.001   & 1.0381 \\
        0.01    & 4.5035 \\
        0.1     & 9.3433 \\
        1       & 8.7925 \\
        10      & -5.3501 \\
        20      & NaN \\
    \end{tabular}
  \end{center}
\end{table}

Influence of the adaptation speed on denoising process.

Adaptation speed Beta describes the rate at which the filter
coefficients are updated in response to changes in the input
signal. If it is too high, the algorithm might “overshoot”
the optimal values causing the filter to become unstable,
leading to bad convergence and poor denoising performance.
If the coefficient is too small however, the filter may not
be able to track the changes in the signal quickly enough to
converge efficiently or at all. This will also lead to poor
results of denoising. In the case of Task1.1,  the best
adaptation speed is around 0.1.

\begin{table}[ht!]
  \begin{center}
    \caption{Improvement of SNR at different values of the input SNR}
        \label{tab:task2tab}
        \begin{tabular}{c|c} 
        \textbf{SNR input} & \textbf{SNR improvement} \\
        \hline
        -5 & 9.4705 \\
        0 & 9.3433 \\
        5 & 8.9773 \\
        10 & 8.0079 \\
        20 & 2.4605 \\
    \end{tabular}
  \end{center}
\end{table}

What happens if the adaptation speed is too high? And too low?

In this part, the improvement of SNR decreased when the
value of $SNR_{in}$ increased. The filter was most efficient
when the initial SNR was negative (-5 dB) and least
efficient for the highest initial SNR (20 dB). This proves
the filter is able to significantly improve the signal even
when it is heavily corrupted by noise – in the case of -5dB.
When the initial SNR is much higher, the improvement is much
smaller, possibly due to the fact the signal has less noise
to begin with.

Best adaptation speed was found to be: 0.011

\newpage
\begin{table}[ht!]
  \begin{center}
      \caption{Improvement of SNR due to adaptive filtering. Echo delay=0.5.}
    \label{tab:task21tab}
    \begin{tabular}{c|c} 
      \textbf{Adaptation speed} & \textbf{SNR improvement} \\
      \hline
        0.001 & 4 \\
        0.01 & 7.6208 \\
        0.1 & 1.6521 \\
        0.05 & 3.7118 \\
        0.005 & 6.8659 \\
        0.007 & 7.3609 \\
        0.009 & 7.5828 \\
        0.011 & 7.6216 \\
        0.012 & 7.5918 \\
    \end{tabular}
  \end{center}
\end{table}

In this experiment we changed the Echo/signal ratio and compared the
improvements in SNR. For very low values of $SNR <= 0.01$ the improvement was
negative, which means the filter has corrupted the noise further. When the
values of the Echo/signal increased, so did the improvement of SNR until the
Echo/signal ratio reached 0.5. For values greater than 0.5, the improvements of
SNR started deteriorating – possibly due to the fact, that for the ratio bigger
than 0.5 the amount of echo was greater than the amount of the original signal,
causing the filter to adapt to the echo rather than the signal. 

\begin{table}[ht!]
  \begin{center}
    \caption{The improvement of signal to echo ratio at different values of echo amplitude}
        \label{tab:task4tab}
        \begin{tabular}{c|c} 
        \textbf{Echo/signal ratio} & \textbf{SNR improvement} \\ \hline
            0.001 & -39.5561 \\
            0.01 & -19.5554 \\
            0.1 & 0.0034 \\
            0.2 & 4.7754 \\
            0.5 & 7.6216 \\
            0.6 & 7.5497 \\
            0.7 & 3.315 \\
            0.99 & 6.2976 \\
    \end{tabular}
  \end{center}
\end{table}

\begin{table}[ht!]
  \begin{center}
      \caption{The influence of the adaptation speed on prediction gain}
        \label{tab:task4tab}
        \begin{tabular}{c|c} 
        \textbf{Echo/signal ratio} & \textbf{Gain} \\ \hline
            0.001 & 0.9433 \\
            0.01 & 4.8289 \\
            0.1 & 7.9827 \\
            1 & 9.2575 \\
            10 & 9.06 \\
            20 & -1297.3 \\
    \end{tabular}
  \end{center}
\end{table}

Describe the influence of the adaptation speed on prediction gain. \\
A faster adaptation speed has significant influence on
prediction gain in adaptive systems. Faster adaptation speed
can lead to higher prediction gain, but too big can lead to
unstable behaviour.

\begin{table}[ht!]
  \begin{center}
      \caption{The influence of the number of coefficients M}
        \label{tab:task4tab}
        \begin{tabular}{c|c} 
        \textbf{no. of prediction coefficients} & \textbf{Gain} \\ \hline
            1 & 7.3939 \\
            2 & 8.4688 \\
            5 & 8.6515 \\
            10 & 9.2575 \\
            20 & 10.2437 \\
            40 & 11.4998 \\
    \end{tabular}
  \end{center}
\end{table}

\begin{table}[ht!]
  \begin{center}
      \caption{Test of the predictor of 1, 2 and 4 coefficients}
        \label{tab:task4tab}
        \begin{tabular}{c|c|c} 
            \textbf{Adaptation speed} & \textbf{no. of prediction coefficients} 
            & \textbf{Gain} \\ \hline
            1 & 1 & 9.1454\\
            5 & 1 & 8.8129\\
            0.5 & 1 & 9.2334\\
            0.1 & 1 & 9.3079\\
            1 & 2 & 69.7873\\
            5 & 2 & 66.1696\\
            0.5 & 2 & 70.0464\\
            0.1 & 2 & 68.2636\\
            1 & 4 & 73.0399\\
            5 & 4 & 60.5352\\
            0.5 & 4 & 73.7804\\
            0.1 & 4 & 74.3916\\
    \end{tabular}
  \end{center}
\end{table}

What is the difference between results obtained for speech signal and sine signal? \\
Because the speech signal is composed of multiple harmonics it is highly
different from simple sine signal. In our experiments we found that the results
obtained for the speech signals and sine signal differ in terms of the
prediction gain and the stability of the predictor. For the speech signals, the
prediction gain generally increases as the adaptation speed (step size) and the
number of prediction coefficients increase, up to a certain point. Beyond that
point, the predictor becomes unstable and the prediction gain decreases
drastically. This behavior is due to the non-stationary nature of speech
signals, which makes it difficult to accurately model the signal using a linear
predictor. On the other hand, for the sine signal, the prediction gain
generally increases as the adaptation speed and the number of prediction
coefficients increase, without any sign of instability. This behavior is due to
the stationary nature of the sine signal, which makes it easier to accurately
model the signal using a linear predictor. Therefore, the speech signals are
generally more challenging to predict accurately than stationary signals like
the sine signal, and require careful selection of the adaptation speed and
number of prediction coefficients to achieve good prediction performance.

\end{document}
